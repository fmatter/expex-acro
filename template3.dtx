% \iffalse meta-comment
%
% Copyright (C) 2022 by Florian Matter
%
% This file may be distributed and/or modified under the
% conditions of the LaTeX Project Public License, either
% version 1.3 of this license or (at your option) any later
% version. The latest version of this license is in:
%
% http://www.latex-project.org/lppl.txt
%
% and version 1.3 or later is part of all distributions of
% LaTeX version 2005/12/01 or later.
%
% \fi
% \iffalse
%<package>\NeedsTeXFormat{LaTeX2e}[1994/06/01]
%<package>\ProvidesPackage{expex-acro}
%<package>    [2023/08/23 v0.0.3 ]
%<*package>
%</package>
%<*driver>
\ProvidesFile{expex-acro.dtx}
\documentclass{ltxdoc}
\usepackage{showexpl}
\lstset{%
  basicstyle=\ttfamily\small,
  commentstyle=\itshape\ttfamily\small,
  showspaces=false,
  showstringspaces=false,
  breaklines=true,
  breakautoindent=true,
  captionpos=t
}
\usepackage{fontspec}
\usepackage[abbrevs=leipzig]{expex-acro}
\lingset{belowglpreambleskip=0ex,aboveglftskip=-0.2ex,aboveexskip=-0.2ex,belowexskip=-1ex}
\usepackage{hyperref}
\EnableCrossrefs
\CodelineIndex
\RecordChanges
\begin{document}
 \DocInput{expex-acro.dtx}
\end{document}
%</driver>
% \fi

%\CheckSum{0}

%\changes{v0.0.3}{2023/08/23}{More versatile exref command in LaTeX label mode; removed underscores from abbreviations.}
%\changes{v0.0.2}{2022/05/17}{Options for predefined abbreviations and exref modes.}
%\changes{v0.0.1}{2022/04/28}{Initial release}

%\GetFileInfo{expex-acro.sty}

%\DoNotIndex{\#,\$,\%,\&,\@,\\,\{,\},\^,\_,\~,\ }
%\DoNotIndex{\@ne}
%\DoNotIndex{\advance,\begingroup,\catcode,\closein}
%\DoNotIndex{\closeout,\day,\def,\edef,\else,\empty,\endgroup}
%\title{The \textsf{expex-acro} package\thanks{This document
%corresponds to \textsf{expex-acro}~\fileversion,
%dated~\filedate.}}
%\author{Florian Matter \\ \texttt{florianmatter@gmail.com}}
%
%\maketitle
%\tableofcontents
%\section{Introduction}
%
%\textsf{expex-acro}, as the name suggests, combines the \textsf{expex} (for typesetting linguistic examples) and \textsf{acro} (for acronyms) packages.
%It provides commands for the following functionalities: referring to examples (Section \ref{sec:exref}), registering and using glossing abbreviations (Section \ref{sec:abbrevs}), and common notations in linguistics (Section \ref{sec:commands}).
%For detailed instructions, refer to the respective documentations of \textsf{expex} and \textsf{acro}.
%
%Below is a multipart example typeset using \textsf{expex} and the |\gl{}| command.
%Note that it uses two distinct labeling mechanisms, the native \textsf{expex} |<tag>| notation, and the \LaTeX{} |\label{tag}| notation.
%Usually, you would use only one of the two in your examples.
% \iffalse
%<*example>
% \fi
\let\orilabel\label
\begin{LTXexample}[preset=\let\label\orilabel]
\pex<ex1> \label{ex:1}
\a<bernese> \label{ex:bernese}
\begingl
\glpreamble Bernese German//
\glpreamble \obj{üsne Vättere}//
\gla üs-ne vätter-e//
\glb our-\gl{dat}.\gl{pl} father.\gl{pl}-\gl{dat}.\gl{pl}//
\glft \qu{to our fathers}//
\endgl
\a<latin> \label{ex:latin}
\begingl
\glpreamble Latin//
\glpreamble \obj{insularum}//
\gla insul-arum  //
\glb island-\gl{gen};\gl{pl}//
\glft \qu{of the islands}//
\endgl
\xe
\end{LTXexample}
% \iffalse
%</example>
% \fi
%
%\section{Referring to examples}\label{sec:exref}
%\DescribeMacro{\exref} The |\exref| command can be used to refer to examples in text.
%The package option |refmode| determines how |\exref| will behave:
% \iffalse
%<*example>
% \fi
To use the built-in tag system (default):
\begin{lstlisting}
\usepackage[refmode=expex]{expex-acro}
\end{lstlisting}
To use \LaTeX{} |\label|s:
\begin{lstlisting}
\usepackage[refmode=latex]{expex-acro}
\end{lstlisting}
% \iffalse
%</example>
% \fi
%\subsection{Using native \texttt{expex} tags}
% \textsf{expex} has its own tag system, e.g.\ |<ex1>| in \exref{ex1} above.
% The corresponding usage is |\exref|\oarg{id1}\marg{id2}.
% Note that when using the \textsf{expex} mechanism, subexamples need to be referred to by their |\pex|'s tag and a period:
% \iffalse
%<*example>
% \fi
\begin{LTXexample}
\exref{ex1}, \exref{ex1.bernese}, and \exref{ex1.latin}.
\end{LTXexample}
% \iffalse
%</example>
% \fi
% If an optional argument is given, |\exref| will assume that these are two examples in a range:
% \iffalse
%<*example>
% \fi
\begin{LTXexample}
\exref[ex1.bernese]{ex1.latin}
\end{LTXexample}
% \iffalse
%</example>
% \fi
%\subsection{Using \LaTeX{} labels}
%\RenewDocumentCommand{\exref}{ o m }{
%        \IfValueTF{#1} {(\ref{#2}#1)}{(\ref{#2})}}
% The alternative is to use the usual \LaTeX{} |\label| commands in your examples, like |\label{ex:1}| in \exref{ex:1} above.
% The corresponding usage is |\exref|\oarg{range}\oarg{id2}\marg{id}.
% In this mode, one can directly refer to subexamples:
% \iffalse
%<*example>
% \fi
\begin{LTXexample}
\exref{ex:1}, \exref{ex:bernese}, and \exref{ex:latin}.
\end{LTXexample}
% \iffalse
%</example>
% \fi
%The first optional argument is simply inserted after the mandatory argument; ranges of subexamples can be done this way.
% \iffalse
%<*example>
% \fi
\begin{LTXexample}
\exref[a--b]{ex:1}
\end{LTXexample}
% \iffalse
%</example>
% \fi
% The second optional argument is interpreted as the start of an example range, ending with the mandatory argument.
%\section{Glossing abbreviations}\label{sec:abbrevs}
%\textsf{expex-acro} loads the \textsf{acro} package for defining glossing abbreviations, displaying them in running text, and printing a list of the ones used.
%\subsection{Package options}
%\DescribeMacro{abbrevs}
% By default, \textsf{expex-acro} loads no pre-defined glossing abbreviations.
% This can be changed by using either
% \iffalse
%<*example>
% \fi
\begin{lstlisting}
\usepackage[abbrevs=leipzig]{expex-acro}
\end{lstlisting}
or
\begin{lstlisting}
\usepackage[abbrevs=all]{expex-acro}
\end{lstlisting}
% \iffalse
%</example>
% \fi
%Note that the current version of \textsf{acro} becomes significantly slower when loading long abbreviation lists (\url{https://github.com/cgnieder/acro/issues/205}).
%Thus, there is a trade-off between having to manually define every glossing abbreviation you use, and speed. 
%\subsection{Commands}
%\DescribeMacro{\newGlossingAbbrev} Define a new glossing abbreviation.
% Usage: |\newGlossingAbbrev|\marg{key}\marg{meaning}.
%Example: |\newGlossingAbbrev{occ}{occultive}|.\\
%\DescribeMacro{\gl} Render a glossing abbreviation.
% Usage: |\gl|\marg{key}.
%Example: |\gl{occ}|.\\
%\DescribeMacro{\glossingAbbrevsList} Print list of used glossing abbreviations.
% Usage: |\glossingAbbrevsList|. Demo:
% \iffalse
%<*example>
% \fi
\begin{LTXexample}
\glossingAbbrevsList
\end{LTXexample}
% \iffalse
%</example>
% \fi
%\section{Semantic commands for linguistic texts}\label{sec:commands}
%An oft-cited advantage of using \LaTeX{} is the separation of content and formatting.
%That principle is broken by using things like |\textit| for object language.
%Of course, italics are a wide-spread standard, but what if you need or want to change how object language is displayed for some reason?
%Apart from |\gl| for glosses, the following commands are built into \textsf{expex-acro}:
% \iffalse
%<*example>
% \fi
\begin{LTXexample}
\begin{enumerate}
\item \obj{xats} (object language)
\item \qu{cat} (free translation)
\item \rc{katta} (reconstructed form)
\item \ort{Chatz} (orthographic representation)
\item \lxm{chatz} (lexeme)
\end{enumerate}
\end{LTXexample}
% \iffalse
%</example>
% \fi
% \StopEventually{^^A
% \PrintChanges
% \PrintIndex}
% \section{Implementation}
<<CODE_CONTENT>>
% \Finale
\endinput